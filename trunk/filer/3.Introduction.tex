\chapter{Introduction}
The project CRIP is, as described in the "itbfis\_course\_note", an interface to help and ease the everyday life of the elderly, chronically ill and their family and care givers. It should be intuitive to use and it should help make information and measurements more reliable, eg. blood pressure measurements, weight measurement  etc. \\
Since this product addresses to the elderly and chronically ill aswell as docters and nurses the product needs to be very easy to learn and use. Since the elderly doesn't have much knowledge of computers and modern technology it is important to make it usable for them. It is also important that docters and nurses can jump right into using the product with very little or no extra training involved and therefore not wasting any time on implementing the product into the everyday work of the nurses and docters.\\

\section{Prototype objective}
The objective of the prototype is to present data in an efficient way so the user will save time and hassle by using the system. The system must be intuitive, easy to learn and remember. A bonus objective for the prototype will be time saving by making it easier and faster to access data, process measurements and logging data. 

\section{Usability goals}
Below is listed the usability goals and a comment for each fitting to this prototype.
\subsection{Effectiveness}
The system should save time for the user. There must be great emphasis on making the users do as few clicks as possible to get to any information or action in the system, making the system time saving for the care givers.

\subsection{Efficiency}
The system should improve the users ability to perform his job. It should give the care givers a great overview of the patients information and data critical for their work. It should also help give the patient an overview of their own medical condition and help give notes from the care givers to family and/or patients.

\subsection{Safety}
Different User levels prevent undesired situations for different users. It should not be possible for the patients to enter and change any data in the system aswell as some information might be hidden for the patient. But this must not be hindrance for the care givers, this should only be a difference in login.

\subsection{Utility}
The product must provide the user with exactly the medical data he needs (also see "Safety"). It is also important to be able to expand the system easily if new tasks are given to different user levels.

\subsection{Learnable}
The product must be intuitive and very easy to learn. Otherwise the user will waste time instead of saving time. Making the system very learnable will also help the elderly since they don't always have the technical expertise needed to use computer and medical systems.

\subsection{Memorable}
The system must be very memorable so user don't have to learn how to use the system every time he/she interfaces with the system. Therefore buttons etc. must always be placed in the same place for all users and at all times.


\section{User experience goals}
\subsection{High Priority}
\begin{itemize}
\item Motivating
\item Engaging
\item Satifying
\item Helpful
\item Enjoyable
\item Pleasureable
\item Enhancing
\item Rewarding
\item Fulfilling
\item Cognitively stimulating
\end{itemize}
The system is suppose to ease the job of the healthcare professional. The person using the system should be able to save time by using the system while finding the system helpful. The system should motivate the professional to do a better job with the information displayed. 

\subsection{Low Priority}
\begin{itemize}
\item Exciting
\item Entertaining
\item Fun
\end{itemize}
The system does not necessarily have to be boring to be useful but fun is not high priority when interfacing with the system.

\subsection{No Priority}
\begin{itemize}
\item Provocative
\item Challenging
\item Surprising
\item Sociability
\item Supporting creativity
\item Emotionally
\end{itemize}
The system might delay the healthcare professional if too challenging or surprising. The professional has no use for a provocative system. 

\section{System overview}
Below is a figure describing the system very briefly.
\begin{figure}[H]
\centering
\includegraphics[width=1\textwidth]{billeder/systemoversigt}
\caption{System overview}
\end{figure}












