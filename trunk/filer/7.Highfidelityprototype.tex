\chapter{High fidelity prototype}


\section{Prototype}
The last iteration of the prototype is the High fidelity prototype. This prototype is as close as it gets to the final product. Therefore the focus of this iteration was to make the prototype look and feel as nice as possible.\\
This prototype is made in a real programming language and in the style which matches the windows store app's.\\
These app's are commonly very simple and very easy to use. This helps achieving our most important usability and user experience goals.\\
Since the app is simple it doesn't over complicate the task at hand by fancy graphics. This makes the app helpful for the caregivers and hopefully also satisfying to use too. With this in mind it achieves most usability goals - effectiveness, efficiency, learnable and memorable. These goals are all achieved when the app is simple and manageable.\\
As it shows with the comparison below, the prototype is simpler and more beautiful than the mid fidelity prototype.\\
\begin{figure}[H]
\centering
\includegraphics[width=.65\textwidth]{billeder/midhigh_compare.png}
\caption{Mid and High fidelity comparison}
\end{figure}
The overall thought behind the prototype is simple and clear buttons with one or two words very precisely describing what it leads to. The philosophy behind this is from the book "Don't make me think"(Steve Krug, 2006). Even though this book is for web development, the points and thought behind are very relevant when developing anything with a graphical user interface.\\
The prototype also doesn't have a whole lot of pages inside pages, the prototype structure is very flat. This would lead to confusion and raise questions from the user like "Where am i now?" or "How do i get to...?". Below is shown a system overview. It shows how few subpages there are. Worth mentioning is that even though all arrows in the "patient login" only goes from "Patient information" and to the other pages, the buttons still stay in the left side and you can with one click get to all the other pages.\\
\begin{figure}[H]
\centering
\includegraphics[width=1\textwidth]{billeder/system_hifi.png}
\caption{System overview}
\end{figure}
For a user (nurse or patient) they simply have to swipe their card (with RFID tag) over the reader next to the screen and they are presented with their "Main Page". In the nurse perspective it is the "Patient information" and in the patient perspective it is the "Se målinger/Udfør målinger" page.\\
The last thing changed in the hifi prototype is the user levels. Since the system should be able to handle patients making measurements themselves we didn't want to present them with trivial/unimportant information. Therefore the patient will be presented with a much simpler interface when signing in with the RFID card. Below is shown the patient main page.\\

\begin{figure}[H]
\centering
\includegraphics[width=.7\textwidth]{billeder/usermainpage_hifi.png}
\caption{Patient main page.}
\end{figure}

The design with the two large buttons/tiles was a result of a discussion with the peer review group.\\
\section{Peer review}
Unfortunately we weren't able to make a peer review with our peer review group. This means the data isn't completely comparable since the first two test was "within subjects" and the last one will be "between subjects". Ideally it should have all been "between subjects" or "within subjects" to be able to compare the results correctly. We asked people on the school at random to make the test-cases on the prototype.

\section{Results of the peer reviews}
This section describes how the data from the review should be handled. Here we have in mind our data might not be completely comparable, but the methods doesn't change whether the data is comparable or not, only the conclusions based upon this data might be wrong. \\
We originally had a "Task 4" but this was discontinued due to a conversation with a real nurse which lead to discarding "Patient journal" from the prototype because a "Patient journal" is all the information about a patient so it made no sense to have it.\\
Below is show a table of "Time-on-task" throughout the project.\\
\begin{table}[H]
\centering
\begin{tabular}{|c|c|c|c|}
\hline 
Values in seconds (s) & Task 1 & Task 2 & Task 3 \\ 
\hline 
• & \multicolumn{3}{c|}{Low fidelity} \\ 
\hline 
Person 1 & 60 & 15 & 9 \\ 
\hline 
Person 2 & 35 & 8 & 6 \\ 
\hline 
Person 3 & 38 & 17 & 7 \\ 
\hline 
• & \multicolumn{3}{c|}{Mid fidelity} \\ 
\hline 
Person 1 & 16 & 10 & 5 \\ 
\hline 
Person 2 & 19 & 7 & 6 \\ 
\hline 
• & \multicolumn{3}{c|}{High fidelity} \\ 
\hline 
Person 1 & 13 & 7 & 5 \\ 
\hline 
Person 2 & 14 & 7 & 5 \\ 
\hline 
Person 3 & 19 & 10 & 6 \\ 
\hline 
\end{tabular} 
\end{table}
Here it is possible to make a t-test to see if we have different stochastic variables. If this is the case we can prove that there has been a significant improvement. This is done in excel.\\


