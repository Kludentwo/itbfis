\chapter{Analysis}
\section{Current state-of-the-art and role-models}
In the development of designing a new product, one must explore the market for similar products to compare the functionalities and flaws to the new product. In Denmark, the health care institutions use the current state of the art software by CSC Vitae Suite to write journals and keep track of visits to patients. This software suite offers many custom solutions to different areas of the health care system, and as the newest product, they’ve launched a solution for handheld devices and tablets. Whether this product is a competing solution to the CRIP, or Vitae can be used to store and organize data from the CRIP system, must be taken in to account in the development. 
The Vitae Suite has the possibility to implement much functionality, and some of them are not relevant to CRIP. What is interesting is that the Danish healthcare system at the moment is implementing the handheld solution to the staff, and because of this it is relevant to see how this device differs to the CRIP. 
The advantages in Vitae Handheld III is according to the manufacture the ability to document unpredictable situations, work with other software including the entire Vitae Suite and finally that the system is constantly being developed with more functions and modules.  \\
However there is some disadvantages too. For example does the user still need to input all journal data and measurements into the system manually, which is very time consuming for the health care staff. Furthermore is one of the main ideas in Crip to give the patient/elderly the opportunity to do measurements by themselves, and this functionality is not a possibility if the Vitae Handheld is a device belonging to the individual doctor/nurse. \\
All in all the conclusion must be, that the Vitae Handheld is a solution worth mentioning and keeping an eye on, because it has the market share and software to implement the features in CRIP. However as the current market seems, there is a market for the CRIP and the new features this brings. \\


\section{Common Usage Scenario}
\subsection{Common usage scenarios}
The most common usage scenario would include medical staff interfacing with the system. The staff would log into the system with their RFID tag. The next course of action would be to perform a measurement. The staff presses the button "Udfør Måling" to do a measurement. This prompts the staff to scan the barcode on the  device which is used to perform the measurement. After the measurement is done the user is prompted to approve or discard the measurement. If the user chooses approve, the measurement results page is shown. If the user chooses discard, the do measurement page is shown.\\
Other common usage scenarios include: Looking at patient information, patient history or doctors comments.

\subsection{Related use cases}

\begin{table}[H]
    \begin{tabular}{|p{4cm}|p{8cm}|}
    \hline
    Name          & Do Measurement "Udfør Måling"   \\\hline
    Actors        & Doctor \\\hline
    Main Scenario & (1) Log Onto System with RFID tag. \\
    ~             & (2) Press "Udfør Måling". \\
    ~             & (3) Scan barcode on wanted device. \\
    ~             & (4a) Approve result \\         
    ~             & (5a) Measurements results are shown.\\\hline
    
    Alternate Scenario & (4b) Discard result \\
        ~             & (5b) go to (3).\\\hline
    \end{tabular}
\end{table}

\begin{table}[H]
    \begin{tabular}{|p{4cm}|p{8cm}|}
    \hline
    Name          & Look at earlier measurements   \\\hline
    Actors        & Nurses and patients \\\hline
    Main Scenario & (1) Log Onto System with RFID tag. \\
    ~             & (2) Press "Se målinger". \\  
    ~             & (3) Earlier measurements is shown.\\\hline
    \end{tabular}
\end{table}

Most use cases following the "Look at earlier measurements " route. Users can press log out at any time. 


\section{User Groups and stakeholders}
\subsection{Primary user group}
The Primary user group for CRIP is the caregivers. They are primarily going to use the system to perform measurements and check data in the system.\\
They are going to need data when they are at the patient so they can treat accordingly or they are doing routine measurements like blood pressure etc.\\


\subsection{Primary stakeholders}
The primary stakeholders are the the public health care sector, since they are going to be the client to the CRIP. They are the ones managing the healthcare provided to the elderly and crippled.\\
Therefore they also want a cheap but effective system, which will help lower wasted time on putting in measurements etc.\\

\subsection{Secondary user group}
The secondary user group is are the elderly and crippled. Since the system is meant to enable patient self measurements they are also going to use the system.\\


\subsection{Secondary stakeholders}
The secondary stakeholders to CRIP might be patients, family to patients and the caregivers. They all have an interest in a reliable and easy-to-use system.\\

\subsection{Peer review methods and analysis of data}
When planning the first peer review we had our usability and user experience goals in mind. The highest general priority in usability is effectiveness/efficiency, which also relates to the user experience goals helpful, fulfilling and enhancing. Therefore we decided measure the prototype improvements mainly on "Time-on-task" and task success. We also felt that low "Time-on-task" shows that the system is efficient, easy to learn and there are few errors when performing a task. We know that conclusions about the other performance metrics cannot be made with data from those we did. But the method is not different when doing only 2 performance metrics than doing 5.\\
Task success was divided into three levels of success:
\begin{itemize}
\item Failure
\item Partly success
\item Complete success
\end{itemize}
These levels will help distinguish how good the success was or even if there was a success.\\
Time-on-task was measured on each test and categorized to find frequencies of the different ranges. The time-on-task is very dependant on the task since complex tasks obviously will take longer time than simple tasks.\\
For each review the tasks needs to be exactly the same to make for a correct comparison.\\

When comparing tasks we look at statistical methods to prove or disprove progress. We assume our data will follow a normal distribution. We write up the hypothesis that the 2 samples are equal. A significance level is chosen at 95\%. This means if the p-value is larger than 0.05 data will not stride against our hypothesis. The p-value is found in Microsoft Excel or a similar spreadsheet software. \\


