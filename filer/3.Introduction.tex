\chapter{Introduction}
The project CRIP is, as described in the "itbfis\_course\_note", an interface to help and ease the everyday life of the elderly, chronically ill and their family and care givers. It should be intuitive to use and it should help make information and measurements more reliable, eg. blood pressure measurements, weight measurement  etc. \\
Since this product addresses to the elderly and chronically ill it is very important to have high affordance (måske det er forkert at skrive at det er affordance måske det bare er "learnability" vi vil have). Since the elderly doesn't have much knowledge of computers and modern technology it is important to make it usable for them as well.\\

\section{Prototype objective}
The objective of the prototype is to present data in an efficient way so the user will save time and hassle by using the system. The system must be intuitive, easy to learn and remember. 

\section{Usability goals}
\subsection{Effectiveness}
The system should save time for the user.

\subsection{Efficiency}
The system should improve the users ability to perform his job.

\subsection{Safety}
Different User levels prevent undesired situations for different users. 

\subsection{Utility}
The product must provide the user with exactly the medical data he needs.

\subsection{Learnable}
The product must be intuitive and very easy to learn. Otherwise the user will waste time instead of saving time.

\subsection{Memorable}
The system must be very memorable so user don't have to learn how to use the system every time he interfaces with the system.


\section{User experience goals}
\subsection{High Priority}
\begin{itemize}
\item Motivating
\item Engaging
\item Satifying
\item Helpful
\item Enjoyable
\item Pleasureable
\item Enhancing
\item Rewarding
\item Fulfilling
\item Cognitively stimulating
\end{itemize}
The system is suppose to ease the job of the healthcare professional. The person using the system should be able to save time by using the system while finding the system helpful. The system should motivate the professional to do a better job with the information displayed. 

\subsection{Low Priority}
\begin{itemize}
\item Exciting
\item Entertaining
\item Fun
\end{itemize}
The system does not necessarily have to be boring to be useful but fun is not high priority when interfacing with the system.

\subsection{No Priority}
\begin{itemize}
\item Provocative
\item Challenging
\item Surprising
\item Sociability
\item Supporting creativity
\item Emotionally
\end{itemize}
The system might delay the healthcare professional if too challenging or surprising. The professional has no use for a provocative system. 












