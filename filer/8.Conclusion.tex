\chapter{Conclusion}

\textit{Conclude with reference to problem formulation}\\

\textit{How is it possible to make it easier, more convenient and faster for caregivers to do routine tasks?}\\
We assume that automating the manual tasks would make it easier, more convenient and faster for caregivers to do routine tasks. A way to prove this would be to observe a caregiver doing routine task and measure the time it takes. The caregiver would then do the tasks on  our prototype while we observe. Afterwards the caregiver will  fill out  a questionnaire about how it feels to use the prototype. \\
Except for the manual task part, we have done this with engineering students from our class. Although a small sample size, the statistics clearly show that the system is easy to use and convenient. 

\textit{Furthermore is it possible to enable patients to make the measurements themselves and also do them reliably?}\\
Horse



\subsection{Discussion}
\textbf{Eye tracking:}\\
If we have had an eye tracker we could have seen which parts of our prototype the user spent most time looking at. This would help improve our prototype in order to remove possible visual noise sources. Alternately we could improve the user interface if we see in the eyes patterns that the person is searching through the interface for a longer period of time. 

\textbf{Test subjects:}\\
In a proper development project we would have chosen a large group of people for user experience tests. Ideally we would have more than 50 people trying complete the tasks we set  out for them to do, while we observe them and measure the time each task takes.  We have had groups  of 3-4 engineering students do tasks in our prototypes. We assume that engineers think alike and therefore have a type of technical know-how. This know-how may or may not exist in the user group that we originally designed the system for.