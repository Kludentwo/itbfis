\chapter{Conclusion}
The system enables for easy handling of routine tasks. This is achieved by a simple prototype without manual inputs, like measurement data and login information. The system has a flat structure making it easy to navigate. They system is designed to have a small learnability curve, minimalistic design and simple functions. Interviews from the reviewers pointed towards this.\\
The different user levels helps to ensure an even simpler interface to the patient which is considered to have the least technical know-how.\\

\section{Discussion}
\textbf{Eye tracking:}\\
If we have had an eye tracker we could have seen which parts of our prototype the user spent most time looking at. This would help improve our prototype in order to remove possible visual noise sources. Alternately we could improve the user interface if we see in the eyes patterns that the person is searching through the interface for a longer period of time. 

\textbf{Test subjects and methods:}\\
In a proper development project we would have chosen a large group of people for user experience tests. Ideally we would have more people trying complete the tasks we set out for them to do, while we observe them and measure the time each task takes.  We have had groups  of 3-4 engineering students do tasks in our prototypes. We assume that engineers think alike and therefore have a type of technical know-how. This know-how may or may not exist in the user group(nurses and patients) that we originally designed the system for.\\
Even though we had minimal amounts of data the methods of analysing the data still applies and would not change with an increase of data. This means we would have easily have up-scaled the reviews and thereby having more reliable data to work with.
